% !TEX root = main.tex
\chapter{The platform}
\label{sec:platform}

The following section details the notation used in the rest of the paper, and the formal definition of paper and researcher attributes (Section \ref{sec:methods}). Moreover we describe the composition of the user interface and all the possibile interactions and visual cues (Section \ref{sec:userinterface}).
\section{Data and notation}
\label{sec:methods} %and goals of ReviewerNet}

The aim of ReviewerNet is to facilitate the reviewer selection process in the academic domain. While designing ReviewerNet, we took into account the characteristics of the \emph{users}, the users' \emph{tasks}, and the \emph{data} the users are working with \cite{MiAi14}. 

The \emph{users} of ReviewerNet are researchers, and in particular those playing the role of journal editors, associate editors, and members of IPCs of big conferences, in any field of research. Their \emph{task} is searching for reviewers for a submitted paper: this involves searching the literature for key papers and authors in the field; evaluating the candidates' research interests and their evolution over time; and assessing the candidates' conflict of interest with respect to the submitting authors and other reviewers. ReviewerNet supports all these subtasks, by visualizing the literature related with a topic, the career of relevant researchers in the field, and the relationships among researchers. 

The data pertain to three types of entities: \emph{papers}, \emph{researchers}, and \emph{citations}. The data attributes are both quantitative and qualitative, and the time dimension is central. 

In ReviewerNet, the attributes of a paper which are visualized are its \emph{citation count} -- the number of papers citing it -- as well as standard \emph{bibliographic attributes} -- title, authors, publication year, venue. Papers are related through \emph{direct citations}. 

Researchers have two attributes in ReviewerNet: relevance and conflict of interest. We define a researcher's \emph{relevance} as a reviewer according to the authorship of relevant papers. The concept of relevance can be tuned according to the user needs (e.g., looking for highly-specialized reviewers, as opposed to generalists). The second attribute of researchers is their \emph{conflict of interest}, with either the submitting authors or other reviewers. We model the conflict of interest after \emph{co-authorship} relations: two researchers have a conflict of interest if they have papers in common. We let the degree of conflict, and hence the availability as a reviewer, be modulated according to the number of papers in common, and the years passed since the last co-authored paper, again according to the user intent. 

Let $\mathcal{P}$ denote the set of papers in a reference dataset, %, with cardinality $\vert \mathcal{P} \vert$, 
and let $\mathcal{P}_{V} \subseteq \mathcal{P}$ be the set of papers relevant to a submission. %, with $\mathcal{P}_{V} \subseteq \mathcal{P}$. The set $\mathcal{P}_{V}$ of relevant papers 
$\mathcal{P}_{V}$ is built by the users starting from a small number of seed papers of their choice (cf. Chapter \ref{sec:demonstration}).%, through the graph expansion functionalities offered by the Paper Network visualization.   

A paper $p \in \mathcal{P}_{V}$ is marked as \emph{selected}, if it is considered as a key paper by the user; we denote by $\mathcal{P}_{S}$ the set of selected papers, with $\mathcal{P}_{S} \subseteq \mathcal{P}_{V} \subseteq \mathcal{P}$. 

If $\mathcal{C}(p)$ is the set of papers citing $p$, the \emph{citation count} $c(p)$ is its cardinality: $c(p) = \vert \mathcal{C}(p) \vert$.

Let $\mathcal{A}(p)$ be the set of authors of a given paper $p$, and $\mathcal{R}$ the set of authors of papers in $\mathcal{P}$. 
%, and $\mathcal{R}_{V}$ be the set of authors of papers in $\mathcal{P}_{V}$: $\mathcal{R}_{V} = \{r \in \mathcal{R} \ s.t. \ \exists \ p \in \mathcal{P}_V : r \in \mathcal{A}(p)\}$. 
Then, the set $\mathcal{R}_C \subseteq \mathcal{R}$ of \emph{candidate reviewers} is given by the set of researchers who authored a selected paper: $$\mathcal{R}_{C} = \{r \in \mathcal{R} \ s.t. \ \exists \ p \in \mathcal{P}_S : r \in \mathcal{A}(p)\}$$ %It holds that $\mathcal{R}_C \subseteq \mathcal{R}_{V} \subseteq \mathcal{R}$.  
%
For a candidate reviewer $r$, let $\mathcal{P}_{S}|_{r}$ be the set of papers in $\mathcal{P}_{S}$ authored by $r$. Then, the \emph{relevance score} $s(r)$ of the candidate reviewer $r$ is defined as a weighted sum of the number of selected and non-selected papers in $\mathcal{P}_V$ authored by $r$: $$s(r) = \alpha \vert \mathcal{P}_{S}|_{r} \vert + \beta \vert \{\mathcal{P}_{V} - \mathcal{P}_{S}\}|_{r}\vert$$ with $\alpha$ and $\beta$ real-valued coefficients summing up to one. We set $\alpha = 0.7$ and $\beta = 0.3$ as default parameters. The set of candidate reviewers will be visualized in the Researcher Timeline in order of their relevance; relevance will also define the dimension of nodes in the Researcher Network.

Finally, $\mathcal{CA}(r)$ denotes the set of co-authors of a researcher $r$, or, in other words, the set of researchers who have a conflict with him/her. 