\chapter{Steganalisi: $\mathbf{\chi}^2$-attack e metodo RS}
Il compito della steganalisi è quello di scoprire l'esistenza di contenuti nascosti e, in alcuni casi, cercare di estrarli o distruggerli \cite{warfare}. Le tecniche di steganalisi possono essere suddivise in due macro categorie:
\begin{itemize}
\item[a.] attacchi basati sull'\textbf{analisi visiva}. Cercano di rilevare la presenza di contenuti nascosti attraverso un'ispezione visiva ad occhio nudo o con l'aiuto di un software, che potrebbe per esempio scomporre l'immagine in bit-plane facilitando notevolmente la ricerca allo steganalista.
\item[b.] metodi di \textbf{analisi statistica}. Sono notevolmente più efficaci rispetto alle tecniche visive. Gli attacchi statistici sono in grado di determinare la presenza di contenuti nascosti con alta precisione e granularità analizzando le alterazioni introdotte dal processo di embedding nelle distribuzioni statistiche dei valori dei pixel delle immagini, spesso impercettibili per l'occhio umano.
\end{itemize}
Le tecniche di steganalisi, che siano visive o statistiche, possono essere ulteriormente classificate in \textit{universali} o \textit{specifiche}. I metodi di steganalisi universale sono pensati per essere applicati a diversi sistemi steganografici, anche sconosciuti. D'altra parte, le tecniche di steganalisi specifica sono progettate per trarre vantaggio dalle falle di sicurezza di uno specifico sistema steganografico e per questo sono più efficienti e affidabili se applicate a stego-oggetti dei quali si conosce a priori la tipologia di embedding \cite{survey2}.\\Come visto nel capitolo precedente, in generale la sicurezza condizionale non implica quella teoretica. Da ciò segue che un sistema steganografico che scambia i valori dei bit meno significativi di alcuni pixel, sebbene possa essere considerato probabilisticamente sicuro, non è necessariamente da considerarsi sicuro in assoluto. Nella restante parte di questo capitolo si vedrà come il processo di embedding dei metodi di LSB-modification lasci \virgolette{tracce} nello stego-oggetto \cite{seeingTheUnseen, warfare} e come sia possibile ricavare informazioni utili da queste tracce con tecniche di steganalisi statistica specifica come il $\mathbf{\chi}^2$-attack \cite{chisq} e il metodo RS \cite{fried1, fried2}.  
		\section{$\mathbf{\chi}^2$-attack}
Il $\mathbf{\chi}^2$-attack è una tecnica steganalitica proposta da Pfitzman e Westfeld che può essere applicata a qualsiasi sistema steganografico in cui delle coppie di valori vengono scambiate fra di loro per incapsulare un messaggio. Queste coppie di valori prendono il nome di \textit{\underline{P}air \underline{o}f \underline{V}alues} (PoV).\\Per meglio comprendere la natura e il significato di queste PoV è sufficiente pensare alla variazione del valore dei pixel successiva alle modifiche introdotte dalla tecnica di embedding LSB. Per esempio, il valore tonale di un pixel $p_1$ pari a 100, successivamente all'incapsulamento potrà ancora essere 100 oppure cambiare in 101, poiché eventualmente solamente il bit meno significativo sarà stato modificato; in modo del tutto analogo, un pixel $p_2$ che vale 101 potrà mantenere il suo valore o cambiarlo in 100. Dunque la coppia $\lbrace 100, 101 \rbrace$ è la PoV relativa a $p_1$ e $p_2$.\\In generale in un'immagine a $n$ bit ci sono $2^n$ possibili valori per un pixel e $2^{n-1}$ possibili PoV. Considerando che gli elementi di una PoV differiscono soltanto nel valore del bit meno significativo, i valori di una generica PoV relativa ad un pixel saranno $(2k, 2k+1)$ con $k$ che varia nell'intervallo $[0, 2^{n-1}-1]$. L'intuizione su cui si basa questo attacco è che, prima dell'incapsulamento di un messaggio, gli LSB possono essere considerati distribuiti in modo omogeneo \cite{chisq, survey}, mentre dopo la procedura di embedding, ovvero dopo che i pixel scelti per nascondere il messaggio avranno cambiato il loro valore da $2k$ a $2k+1$ o viceversa, le frequenze dei due valori di ogni PoV tenderanno ad avvicinarsi di più fra di loro rispetto a quanto non succedesse nella cover \cite{chisq}. La figura \ref{hist} mostra a sinistra le frequenze dei valori dei pixel di una cover in scala di grigi, mentre a destra vi sono le frequenze dei valori di grigio dopo l'incapsulamento di un messaggio.\\$\mathrm{\grave{E}}$ facile notare che, nell'istogramma relativo allo stego-oggetto, le coppie di colonne relative alle PoV tendono ad avere altezza molto più simile rispetto a quanto non succeda nella cover.\\Questo fenomeno sarà tanto più accentuato quanto più i bit del messaggio saranno distribuiti uniformemente sui pixel della cover \cite{survey}.  

L'idea dell'attacco statistico è dunque quella di usare il test del $\chi^2$ per confrontare la distribuzione delle frequenze dei valori delle PoV attesa $O_e$ con quella realmente osservata e misurata nel potenziale stego-oggetto.\\Qualora il lettore necessitasse di chiarimenti ed approfondimenti sul test del chi-quadro e le sue applicazioni può fare riferimento all'appendice alla fine dell'elaborato.\\Il punto cruciale del $\chi^2$-attack è il calcolo della distribuzione teorica $O_e$.\newpage \noindent Se lo steganalista ha a disposizione un campione di potenziali stego-oggetti non potrà estrarre $O_e$ né direttamente dal campione, considerando che al suo interno potrebbero essere state effettuate modifiche steganografiche, né dalle immagini originali, poiché spesso non sono in suo possesso.\\Siano quindi $O_{2n}$ e $O_{2n+1}$ le occorrenze dei valori di una PoV $(2n, 2n +1)$, si può notare che la loro somma $O_{2n} + O_{2n+1}$ rimane costante; ciò accade sia poiché la distribuzione dei valori pari e dispari è da considerarsi uniforme \cite{survey2}, sia perché la funzione di incapsulamento sovrascrive solamente i bit meno significativi, scambiando di fatto i valori dispari con i pari e viceversa. La conseguenza di questa proprietà statistica è che la media aritmetica delle occorrenze dei valori delle PoV è la stessa sia nella cover che nello stego-oggetto. Ciò consente quindi allo steganalista di calcolare $O_{e_i}$, ovvero la frequenza attesa per l'i-esima coppia di valori, come media aritmetica di $O_{2i}$ e $O_{2i+1}$:
\[O_{e_i} \: = \: \frac{O_{2i} + O_{2i+1}}{2} \quad \forall \, i \in [0, 2^{n-1}-1]\]
$\mathrm{\grave{E}}$ inoltre interessante notare come per calcolare la distribuzione della frequenza attesa non sia necessario disporre né della cover originale né di qualche altra informazione esterna, ma sia sufficiente essere in possesso del potenziale stego-oggetto.
\\La statistica del $\chi^2$ a $k$ gradi di libertà sarà quindi data da 
\[\chi^{2}_{k} \: = \: \sum_{i=0}^{l} \frac{(O_{2i} - O_{e_i})^2}{ O_{e_i}}\]
dove $l =  2^{n-1}-1$ e $k = l \, - \, 3 $.
\\Di conseguenza, sia $\Gamma(\cdot)$ la funzione gamma di Eulero, la probabilità $p$ che all'interno di una cover sia stato incapsulato un messaggio è calcolata come:
\[p =  1 - \frac{1}{2^{\frac{k}{2}}\Gamma (\frac{k}{2})}\int_{0}^{\chi^{2}_{k}}e^{-\frac{x}{2}}x^{\frac{k}{2}-1}\, dx \]
\subsection{Risultati sperimentali}
Il $\chi^2$-attack mostra ottimi risultati se applicato a messaggi incapsulati in modo sequenziale, ovvero inserendo i bit del messaggio nell'ordine in cui verrebbero normalmente letti \cite{chisq}. La figura 4.2 mostra un'evidenza sperimentale di quanto appena detto: all'interno dell'immagine a sinistra è stato incapsulato un messaggio di $3 \,600$ byte, mentre il grafico sulla destra mostra la variazione del valore calcolato per $p$ analizzando lo stego-oggetto in modo incrementale, da sinistra verso destra e dall'alto verso il basso: è facile notare come tale valore sia dapprima vicino ad 1, mentre tenda a 0 all'aumentare della dimensione della regione di pixel considerata \cite{chisq}. Ciò avviene poiché, una volta che i pixel contenenti il messaggio, situati all'inizio dell'immagine, sono stati visitati, il campione utilizzato per il calcolo di $p$ conterrà un numero di pixel inalterati tale da far diminuire in modo statisticamente significativo la differenza fra le frequenze teoriche delle PoV e quelle osservate, permettendo così allo steganalista di determinare con precisione sia il punto in cui ha inizio l'incapsulamento del messaggio, sia quello in cui termina.

Inizialmente il campione comprende l'$1\%$ dei pixel dell'immagine partendo dal bordo in alto a sinistra, e la probabilità di embedding calcolata per quella porzione di immagine è $p = 0.8826$. Si evince chiaramente che finché si analizza la metà superiore dell'immagine il valore di $p$ non scende sotto il $77\%$. All'aumentare della dimensione del campione, quando si supera la soglia del $52\%$ in modo da includere anche i pixel della parte inferiore dell'immagine, il valore calcolato per $p$ scende drasticamente fino ad arrivare a $0$ e ciò avviene a causa della dimensione del messaggio nascosto.\\Calcolando in modo incrementale il valore di $p$ lo steganalista sarà quindi in grado non solo di determinare se un messaggio è stato nascosto con probabilità molto alta, ma potrà anche stimarne con molta precisione la dimensione.\\Se i pixel in cui il messaggio verrà incapsulato sono invece scelti in modo pseudo-casuale il test perde notevolmente di efficacia a meno che il messaggio non occupi la maggior parte dei pixel \cite{fried1}.
		\section{Metodo RS}
I metodi statistici basati sul conteggio dei campioni statistici, come quello della sezione precedente, trascurano un notevole numero di informazioni utili ricavabili direttamente dal potenziale stego-oggetto. L'intuizione di Fridrich e Goljan \cite{fried1, fried2} per superare il problema della rilevazione di messaggi incapsulati in modo pseudo-casuale è quella di sfruttare le correlazioni spaziali fra i pixel dello stego-oggetto. Il punto cruciale della loro analisi è la scoperta e quantificazione di queste relazioni fra pixel. Una volta definita la relazione e una sua metrica, lo steganalista potrà analizzarla e studiare come questa misura cambia dopo un'operazione di embedding.\\Si assuma che le cover siano immagini con $M\times N$ pixel e che i valori dei pixel appartengano a $P=[0, 2^b-1]$, con $b$ profondità in bit del colore. Il primo passo del metodo è quello di suddividere lo stego-oggetto in sottoinsiemi disgiunti di $n$ pixel adiacenti.\\La correlazione spaziale fra i pixel sarà quindi definita mediante una funzione discriminante $f$ che, dato un sottoinsieme di pixel $G = \lbrace x_1, x_2,\, \dots, x_n \rbrace$, restituisce un numero reale che misura l'omogeneità di $G$ \cite{fried1, fried2}. La funzione $f$ è definita come segue:
\[ f(x_1, x_2,\, \dots, x_n)=\sum^{n-1}_{i = 1}\vert x_{i+1} - x_i \vert\]
Il valore calcolato dalla funzione discriminante $f$ sarà quindi tanto più alto quanto più $x_1, x_2,\, \dots, x_n$ è disomogeneo.\\L'utilità della metrica definita sopra segue dal fatto che l'embedding mediante LSB-modification aumenta il rumore e la disomogeneità all'interno dell'immagine \cite{fried1}, portando quindi ad un incremento del valore calcolato da $f$ per uno stesso gruppo di pixel.\\Dopo aver stabilito una misura per le correlazioni spaziali fra i pixel di un'immagine, bisogna poter disporre di una definizione del processo di incapsulamento LSB che si presti ad essere utilizzata in un'analisi statistica.\\Nella sezione 4.1 è stato fatto vedere come il processo di embedding LSB non faccia altro che mappare il valore di un pixel su una coppia di valori.\\$\mathrm{\grave{E}}$ dunque ragionevole descrivere la procedura di LSB-modification come una funzione di \textit{flipping} $F_1$ che, dato il valore di un pixel $\in P$, lo scambia con uno degli elementi della PoV a cui appartiene. Per un'immagine ad 8 bit si avranno dunque le seguenti associazioni: $0\leftrightarrow 1,\, 2 \leftrightarrow 3,\, \dots,\, 254\leftrightarrow 255$. La figura 4.3 fornisce un ulteriore esempio in forma tabellare di tutti i possibili \textit{flip} del valore di un pixel con 8 bit di profondità tonale.

Conseguentemente si potrà definire l'operazione duale di \textit{shifting} $F_{-1}$ come $-1\leftrightarrow 0,\, 1 \leftrightarrow 2,\, \dots,\, 255\leftrightarrow 256$. $F_1$ e $F_{-1}$ sono quindi tali che:
\begin{equation}
\label{1}
\forall x\in P, \; F_{-1}(x) = F_1 (x+1) -1
\end{equation} 
Per completezza è anche possibile definire l'operazione di \textit{identità} $F_0$ con $F_0 (x) = x$. Poiché in generale si vogliono applicare diverse $F_i$ a diversi pixel è conveniente definire il procedimento di embedding come l'applicazione di una \textit{maschera} $M$ di funzioni di flipping, dove $M$ è un'ennupla con valori -1, 0 e 1. A questo punto si nota che la funzione discriminante $f$ e le funzioni di flipping $F_i$ definiscono tre tipi di insiemi di pixel, che differiscono fra loro nel modo in cui l'applicazione di una maschera di flipping $M$ cambia il valore di $f$ calcolato sull'insieme stesso.
Sia dunque $G$ un insieme di pixel, allora
\begin{itemize}
\item[] $G$ è di tipo $R$ (\textit{\underline{r}egular}) $ \quad \; \; \;  \Leftrightarrow \quad f(F_M(G)) > f(G)$
\item[] $G$ è di tipo $S$ (\textit{\underline{s}ingular}) $\, \, \, \, \;  \; \; \Leftrightarrow \quad f(F_M(G)) < f(G)$
\item[] $G$ è di tipo $U$ (\textit{\underline{u}nchanged}) $\, \,\Leftrightarrow \quad f(F_M(G)) = f(G)$
\end{itemize}   
dove $F_M(G) = (F_{M_1}(x_1), F_{M_2}(x_2), \, \dots, F_{M_n}(x_n))$. La tecnica di steganalisi che verrà ora presentata si basa sull'analisi della variazione del numero di insiemi di pixel regolari e singolari all'aumentare della lunghezza del messaggio nascosto nell'LSB-plane \cite{fried2}.\\Siano $R_M$ e $S_M$ rispettivamente la frazione di insiemi di pixel regolari e singolari per la maschera $M$ rispetto al numero totale degli insiemi. Sia inoltre $-M$ la maschera \textit{opposta} di $M$, ovvero tale che $\forall i$:
\begin{itemize}
\item se $M_i = 1$ allora $-M_i = -1$;
\item se $M_i = -1$ allora $-M_i = 1$;
\item se $M_i = 0$ allora $-M_i = M_i$
\end{itemize} 
$\mathrm{\grave{E}}$ utile notare che, per una qualsiasi maschera $M$, la somma del numero di insiemi dei pixel di tipo $R$, $S$ e $U$ è uguale alla somma di tutti gli insiemi di pixel che compongono una generica immagine o, in formule: \[R_M + S_M + U_M = 1\]\\Le seguenti disequazioni sono dunque sempre soddisfatte:
\[R_M + S_M \leq 1\]
\[R_{-M} + S_{-M} \leq 1\]
Le definizioni e proprietà esposte fin'ora consentono di formulare l'ipotesi statistica alla base del metodo di Fridrich: per una generica cover il valore di $R_M$ è approssimativamente uguale al valore di $R_{-M}$ e lo stesso vale per $S_{M}$ e $S_{-M}$ \cite{fried1, fried2}:
\begin{equation}
\label{hp}
R_{M} \cong R_{-M} \wedge S_{M} \cong S_{-M} 
\end{equation}
Una dimostrazione euristica dell'ipotesi segue da un'analisi dell'equazione \ref{1}. Rileggendo la formula risulta infatti immediato che applicare $F_{-1}$ ad un'immagine equivale ad applicare $F_1$ alla stessa immagine i cui valori tonali sono stati incrementati di una unità. Segue quindi che, essendo la funzione discriminante $f$ una metrica dell'omogeneità di un gruppo di pixel, un incremento uniforme del valore di tutti i pixel non ha alcuna influenza sul numero di insiemi regolari e singolari in un'immagine \cite{fried1, fried2}; ovvero in una cover, nella quale per ipotesi gli LSB sono distribuiti in modo uniforme \cite{fried1, fried2, chisq, survey, survey2}, l'applicazione di $M$ o $-M$ non cambia in modo statisticamente significativo i valori di $R$ ed $S$.
L'ipotesi \ref{hp} è stata inoltre provata sperimentalmente sia su immagini acquisite con una fotocamera in vari formati, sia su immagini che avevano subito \textit{post-processing} \cite{fried2}.\\L'unico caso in cui la congettura \ref{hp} non risulta essere valida è in presenza di LSB-plane randomizzati, come può essere quello di un'immagine all'interno della quale è stato incapsulato un messaggio in modo pseudo-casuale mediante tecniche di LSB-modification \cite{fried2}. Si nota infatti che la randomizzazione dell'insieme dei bit meno significativi di un'immagine fa tendere la differenza fra $R_M$ e $S_M$ a 0 all'aumentare della dimensione del messaggio incapsulato \cite{fried1, fried2}. Si otterrà così che $R_M \cong S_M$, dopo aver applicato una funzione di flipping agli LSB di circa la metà dei pixel -che è quello che succede in media dopo l'embedding di un messaggio tramite LSB-modification \cite{seeingTheUnseen, warfare}. L'opposto accade ad $R_{-M}$ e $S_{-M}$, in quanto la loro differenza tende ad aumentare nelle condizioni sopra specificate.\\$\mathrm{\grave{E}}$ ora possibile definire lo strumento principale di questo metodo di steganalisi statistica specifica: il digramma RS.

Il diagramma RS in figura 4.3 mostra $R_{M}$, $R_{-M}$, $S_{M}$, e $S_{-M}$ come funzioni del numero di pixel a cui è stata applicata una funzione di flipping.\\L'obiettivo dello steganalista sarà dunque quello di fornire una stima delle quattro curve che compongono il diagramma e delle loro intersezioni, per poter poi determinare la lunghezza dell'eventuale messaggio nascosto che, come si vedrà in seguito, risulta essere funzione proprio delle intersezioni fra le curve \cite{fried1, fried2}. Fridrich e Goljan hanno raccolto prove sperimentali che dimostrano come le curve che descrivono $R_{-M}$ e $S_{-M}$ siano modellabili da rette, mentre le curve riferite a $R_{M}$ e $S_{M}$ risultano essere ben approssimate da polinomi di secondo grado, quindi parabole. Di seguito verrà mostrato come calcolare i parametri che definiscono le curve -e quindi la lunghezza del messaggio nascosto- utilizzando i punti evidenziati in figura 4.3.\\Sia $s$ uno stego-oggetto ed $m$ un messaggio di lunghezza $\ell$ non nota, espressa in percentuale di pixel. Se $m$ è stato nascosto all'interno di $s$ con una procedura di LSB-modification che ha cambiato il valore di pixel scelti in modo pseudo-casuale, allora:
\begin{itemize}
\item[a.]calcolando il numero di insiemi di pixel regolari e singolari su $s$ si otterranno le stime statistiche di $R_{M}(\ell /2)$, $R_{-M}(\ell /2)$, $S_{M}(\ell /2)$ e $S_{-M}(\ell /2)$. Il fattore $\frac{1}{2}$ è conseguenza dell'assunzione secondo la quale, in un generico stego-oggetto creato con LSB-modification, la funzione di flipping è applicata in media alla metà dei pixel \cite{fried1, seeingTheUnseen, warfare};
\item[b.]se si cambia il valore del bit meno significativo di \textit{tutti} i pixel e si ricalcolano i valori di $R$ ed $S$ i valori in uscita saranno quelli relativi ai punti $R_{M}(1-\ell /2)$, $R_{-M}(1-\ell /2)$, $S_{M}(1-\ell /2)$ e $S_{-M}(1- \ell /2)$ \cite{fried1, fried2};
\item[c.]infine i valori dei punti $R_{M}(1/2)$ e $S_{M}(1/2)$ si ottengono ridisponendo in modo pseudo-casuale i bit meno significativi dell'immagine. Se si vuole applicare questa strategia, per avere una stima statisticamente affidabile, occorrerà però effettuare i calcoli più volte, poiché i valori dei due punti di cui sopra dipendono strettamente dal riposizionamento adottato \cite{fried1}.
\end{itemize}
Si potranno quindi definire le rette mediante il numero di insiemi regolari e singolari calcolati in $\ell /2$ e $1-\ell /2$ (punti a e b), mentre le due parabole saranno determinate dai punti rimanenti.\\$\mathrm{\grave{E}}$ però possibile rendere più efficiente il processo e, contemporaneamente, aumentare la precisione della stima della lunghezza $\ell$ del messaggio.\\ Senza effettuare direttamente i calcoli descritti nel punto c, si possono infatti ricavare i valori di $R_{M}(1/2)$ e $S_{M}(1/2)$ considerando che:
\begin{enumerate}
\item il punto di intersezione delle curve $R_{M}$ e $R_{-M}$ ha la stessa ascissa del punto di intersezione delle curve $S_{M}$ e $S_{-M}$;
\item le curve $R_{M}$ e $S_{M}$ si intersecano nel punto $x=50\%$ o, in modo del tutto equivalente, $R_{M}(1/2)=S_{M}(1/2)$.
\end{enumerate} 
La prima osservazione è essenzialmente una versione più forte dell'ipotesi \ref{hp}, la seconda osservazione è invece conseguenza del fatto che cambiando il valore del bit meno significativo di circa il 50\% dei pixel la differenza fra $R_{M}$ e $S_{M}$ tende a zero \cite{fried1, fried2, seeingTheUnseen, warfare}.\\
Le osservazioni appena descritte conducono quindi non solo ad un consistente risparmio di risorse computazionali, ma permettono anche di avere una stima più precisa di $\ell$, in quanto possono essere usate come ulteriori vincoli nella derivazione delle equazioni delle parabole che descrivono $R_{M}$ e $S_{M}$ \cite{fried1}.\\Dopo aver normalizzato l'asse delle ascisse in modo che $\ell/2$ corrisponda allo $0$ e $1-\ell/2$ corrisponda ad $1$ con la sostituzione $z= (x-\frac{\ell}{2})/(1-\ell)$, l'ascissa del punto di intersezione fra $R_{M}$ e $R_{-M}$ ed $S_{M}$ e $S_{-M}$ sarà quindi calcolata come radice della seguente equazione quadratica in $z$ \cite{fried1, fried2, improved}:
\[2(d_1+d_0)z^2 + (d_{-0}-d_{-1}-d_{1}-3d_{0})z + d_{0} - d_{-0} = 0 \quad \mathit{dove}\]
$\qquad d_{0}=R_{M}(\ell /2)-S_{M}(\ell /2)\qquad d_{1}=R_{M}(1-\ell /2)-S_{M}(1-\ell /2)\\
\; d_{-0}=R_{-M}(\ell /2)-S_{-M}(\ell /2)\qquad d_{-1}=R_{-M}(1-\ell /2)-S_{-M}(1-\ell /2)$\vspace{0.4cm}\\
La lunghezza $\ell$ del messaggio sarà poi calcolata in funzione della radice $z_i$ minore in valore assoluto \cite{fried1}:
\[\ell = \frac{z_i}{z_i -\frac{1}{2}}\]
\subsection{Accuratezza del modello}
La precisione della stima della lunghezza del messaggio nascosto nel metodo RS è influenzata da tre fattori principali:
\begin{itemize}
\item \textbf{Assunzione iniziale:} la steganalisi RS potrebbe rilevare una valore non nullo per la lunghezza $\ell$ del messaggio anche su alcune cover. Questo errore sperimentale porta lo steganalista a dover accettare un'\textit{ipotesi non nulla iniziale} e fissa un limite sulla precisione del metodo RS. Sono state raccolte evidenze empiriche per stimare la precisione della steganalisi RS ed è emerso che, su un campione di 331 immagini in scala di grigio testate, l'ipotesi non nulla assume distribuzione Gaussiana con deviazione standard pari allo $0.5\%$ della lunghezza del messaggio espressa in percentuale sulla dimensione dell'immagine \cite{fried1}.\\$\mathrm{\grave{E}}$ stato inoltre osservato che la varianza tende ad aumentare per immagini piccole, rumorose o a colori, mentre decresce, tendendo a 0, per immagini JPEG, sottoposte a post-processing, scannerizzate, o acquisite tramite una fotocamera digitale \cite{fried1, fried2}. 
%Per immagini piccole o con un alto tasso di rumore o anche per immagini a colori, la varianza tende ad aumentare; l'ipotesi e la sua varianza tendono invece a zero per immagini JPEG, immagini sottoposte a post-processing, scannerizzate, o acquisite tramite una fotocamera digitale \cite{fried1, fried2}. 
\item \textbf{Rumore:} poiché per immagini con un alto tasso di rumore la differenza fra $R$ ed $S$ tende a 0, le curve del diagramma RS si intersecheranno con un angolo di incidenza molto piccolo. Si avrà come conseguenza una diminuzione della precisione del metodo. Lo stesso vale per le immagini a bassa qualità o sovracompresse \cite{fried1}.
\item \textbf{Inserimento del messaggio:} il metodo RS porta a risultati più precisi ed affidabili se applicato a stego-oggetti con messaggi nascosti in modo pseudo-casuale. Per i messaggi incapsulati in modo sequenziale è preferibile l'impiego del metodo proposto da Westfeld e Pfitzman \cite{chisq}.
\end{itemize}
\subsection{Risultati sperimentali}
Fridrich e Goljan hanno effettuato dei test per dimostrare l'efficacia del metodo proposto. Il risultato di questi test è mostrato in figura 4.4.\\Per l'esperimento è stata utilizzata un'immagine $1536\times 1024$ a colori acquisita con una fotocamera digitale Kodak DC260. L'immagine è stata quindi convertita in scala di grigi e ridimensionata a $384\times 256$ pixel prima di essere usata come cover. Sono poi state create una serie di stego-immagini da incapsulare nella cover applicando una funzione di flipping al valore tonale di alcuni pixel scelti in modo pseudo-casuale. Il numero di pixel scelti variava dallo 0\% al 100\% dei pixel che compongono l'immagine.\\Nella figura la linea solida rappresenta la percentuale di pixel alterati rilevata col metodo RS, mentre la linea tratteggiata corrisponde al numero di pixel realmente usati per l'incapsulamento. La parte inferiore dell'immagine è un ingrandimento che mostra l'andamento dell'errore di rilevazione: è facile notare come la steganalisi RS sia caratterizzata da un alto grado di precisione nella stima della lunghezza del messaggio nascosto, in quanto l'errore fra il numero reale di pixel alterati e quello calcolato col metodo RS è quasi sempre minore dell'1\% del numero totale di pixel analizzati \cite{fried1, fried2, survey}.
%\newpage
\noindent Le performance del metodo RS sono anche state testate su stego-immagini ottenute con i software steganografici attualmente disponibili. La cover usata per questo test è una fotografia a 24 bit, memorizzata in formato JPEG, scattata da una Kodak DC260 e ricampionata da $1536\times 1024$ pixel a $1024\times 744$ pixel. Il messaggio incapsulato aveva una lunghezza in pixel pari al 5\% dei pixel totali dell'immagine e la tecnica di embedding usata è stata in ogni esperimento quella dell'LSB-modification. I risultati, ancora una volta molto positivi, sono mostrati nella tabella 4.1.
\begin{table}[!h]
\centering
\label{tab}
\begin{tabular}{|llll|}
\hline
\textbf{Software}         & \textbf{Rosso (\%)} & \textbf{Verde (\%)} & \textbf{Blu (\%)} \\ \hline
\textit{Ipotesi iniziale} & 0.00 (0.00)         & 0.17 (0.00)         & 0.33 (0.00)       \\ \hline
\textbf{Steganos}         & 2.41 (2.44)         & 2.70 (2.46)         & 2.78 (2.49)       \\
\textbf{S-Tools}          & 2.45 (2.45)         & 2.62 (2.43)         & 2.75 (2.44)       \\
\textbf{Hide4PGP}         & 2.44 (2.46)         & 2.62 (2.46)         & 2.85 (2.45)       \\ \hline
\end{tabular}
\caption{Stima dell'ipotesi non nulla iniziale e della percentuale di pixel usati per l'incapsulamento calcolati col metodo RS. Le percentuali indicate in parentesi rappresentano il dato reale, mentre quelle fuori dalle parentesi sono le percentuali rilevate.}
\end{table}