\chapter{Conclusioni e approfondimenti}
L'obiettivo principale della trattazione era quello di analizzare nel dettaglio una parte del vastissimo scenario dell'information hiding odierno. A questo proposito, dopo aver presentato il contesto del problema, si è cercato di introdurre il lettore al problema della definizione della sicurezza per un generico sistema steganografico. Si è visto come, sebbene la teoria dell'informazione suggerisca vari parametri e modelli per definire la sicurezza steganografica, nessuno di questi parametri e modelli potesse essere realmente applicato ad alcun caso pratico \cite{sec, warfare}. $\mathrm{\grave{E}}$ stata quindi formulata una definizione di sicurezza probabilistica \cite{sec} che, imponendo vincoli \textit{ragionevolmente} meno stringenti sulla natura e struttura del sistema rispetto alle definizioni di matrice teoretica, può essere facilmente applicata e verificata.\\Successivamente si è fornita una definizione di steganografia e delle principali metodologie impiegate per incapsulare messaggi nascosti in file multimediali. L'attenzione si è quindi spostata su un metodo di steganografia spaziale: l'LSB-modification. La tecnica è stata presentata nel dettaglio e sono stati forniti esempi pratici di applicazione che ne hanno sottolineato la facilità d'impiego in casi reali \cite{survey, survey2, seeingTheUnseen, warfare}. $\mathrm{\grave{E}}$ stato poi dimostrato formalmente che un sistema steganografico che usi l'LSB-modification come tecnica di embedding è da considerarsi condizionalmente sicuro \cite{sec}.\\Successivamente, sottolineando che la sicurezza probabilistica non implica necessariamente quella teoretica \cite{sec}, è stata introdotta la steganalisi in generale e sono stati analizzati nel dettaglio due metodi di steganalisi statistica specifica applicabili a generici stego-oggetti derivanti da processi di LSB-modification.
Per entrambi i metodi è stato mostrato come, essendo solo in possesso dello stego-oggetto, quindi senza nessun'altra informazione esterna, sia possibile ricavare informazioni utili quali probabilità di embedding e lunghezza del messaggio nascosto \cite{fried1, fried2, chisq, survey, survey2}.
In particolare si è visto come il $\chi ^2$-attack sia più efficiente ed affidabile se applicato a stego-oggetti con messaggi incapsulati in modo sequenziale \cite{chisq}, e come il metodo RS risulti essere più performante su messaggi incapsulati in modo pseudo-casuale all'interno dei pixel della cover scelta \cite{fried1, fried2}.\\In definitiva, pur non avendolo dimostrato formalmente, sarà chiaro al lettore che la \virgolette{battaglia} fra la steganografia e la steganalisi non avrà forse mai fine. Ogniqualvolta verrà proposto un nuovo sofisticato metodo steganografico sarà necessario raffinare in modo altrettanto sofisticato le tecniche di steganalisi esistenti, se non crearne delle nuove. Tutto ciò continuerà ad accadere fintantoché non si arriverà quantomeno ad una definizione completa ed applicabile di sicurezza steganografica; il che rappresenta un problema ancora aperto per la ricerca e di non facile soluzione \cite{sec, seeingTheUnseen, warfare}.\\Verrà ora fornito al lettore un elenco dei principali approfondimenti riguardanti gli argomenti discussi e spunti di ricerca futuri nel settore dell'information hiding con relativi riferimenti:  
		\begin{itemize}
		\item ogni metodo di steganografia sembra avere un \textit{limite superiore} sulla lunghezza del messaggio da incapsulare chiamato \textit{capacità steganografica}. Questa misura fornisce una stima del numero di bit alterabili senza che un osservatore esterno possa accorgersene \cite{fried1}. Determinare la capacità steganografica è un compito molto difficile anche per metodi semplici. Chandramouli et al. \cite{sup} hanno però dimostrato che, per le tecniche di LSB-modification, tale soglia dipende strettamente dal metodo steganalitico impiegato e quindi dal \textit{massimo errore} che si compie nella stima della lunghezza $\ell$ del messaggio incapsulato;
		\item Provos \cite{provos} ha esteso il $\chi^2$\textit{-attack} aumentandone l'efficacia e l'affidabilità anche se usato su messaggi incapsulati in modo pseudo-casuale all'interno dell'LSB-plane di uno stego-oggetto;
		\item per rendere la steganografia LSB più robusta ad attacchi basati sulle PoV è stato proposto un metodo, detto di LSB-\textit{matching} \cite{mat1, mat2}.\\Invece di sostituire completamente gli LSB dei pixel della cover, con questa tecnica si aggiunge o sottrae 1 al valore del pixel qualora non dovesse coincidere coi bit del messaggio. Così facendo le distribuzioni dei valori delle PoV non subiscono alterazioni evidenti come nel caso dell'LSB-modification, rendendo così questo tipo di LSB embedding resistente agli attacchi statistici.
		\item in data 06/12/2016 è stato pubblicato dalla ESET, un'azienda slovacca che si occupa di sicurezza informatica, un report che documenta la diffusione di un malware denominato Stegano \cite{sito}. Stegano infetterebbe i dispositivi servendosi di un \textit{banner} pubblicitario apparentemente innocuo. L'annuncio, nella sua versione \virgolette{avvelenata}, nasconde infatti all'interno del canale alfa -quello che regola la trasparenza dei pixel- uno script in grado di reindirizzare l'utente che lo visualizza su una pagina web infetta. Gli utenti colpiti dallo stego-malware sarebbero circa un milione secondo le stime dei ricercatori.
		\end{itemize}
Alla luce di tutte le osservazioni fatte, il panorama dell'information hiding si configura come uno fra i più vasti della sicurezza informatica \cite{warfare} e sempre più tecniche ed accorgimenti vengono studiati e presentati.\\La tendenza è quella di progettare metodi, sia steganografici che steganalitici, che traggano vantaggio da entrambe le parti coinvolte nella \virgolette{competizione} \cite{survey, survey2, warfare}. Spesso succede infatti che i metodi steganografici vengano ideati con l'obiettivo di preservare importanti proprietà statistiche delle immagini, mentre gli steganalisti provino a definire nuove misure discriminanti.\\In sostanza sembra proprio che Alice non smetterà mai di provare a comunicare in modo sicuro ed invisibile con Bob finché ne ha l'occasione, mentre Wendy continuerà ad ingegnarsi per assicurarsi che nessun messaggio segreto venga scambiato fra i due senza che lei possa notarlo.\\
Qualora il lettore volesse avere una visione d'insieme più completa ed approfondita di tutte le tecniche steganografiche e steganalitiche attualmente disponibili si rimanda alle rassegne \cite{survey, survey2, warfare}.
\appendix