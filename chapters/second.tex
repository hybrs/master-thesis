	\chapter{Definizione di sistema steganografico sicuro}
		La sicurezza di una comunicazione steganografica fra due entità risiede nell'impossibilità per un osservatore esterno di distinguere fra stego-oggetti e oggetti senza alcun contenuto nascosto. Il \textit{problema dei prigionieri} introdotto da Simmons \cite{simmons} aiuta a chiarire questo concetto. Si supponga che due personaggi fittizi Alice e Bob vengano arrestati e posti in due celle separate. Per poter mettere a punto un piano di fuga hanno bisogno di comunicare, ma purtroppo la comunicazione è controllata da una guardia, Wendy. Qualora Wendy notasse qualcosa di sospetto nella loro comunicazione la interromperebbe. Lo scambio di messaggi fra Alice e Bob può dunque essere modellato come una comunicazione steganografica il cui obiettivo è quello di scambiare messaggi senza sollevare alcun sospetto. Gli elementi principali di una comunicazione di questo tipo sono la \textit{cover}, il messaggio che si vuole inviare e lo \textit{stego-oggetto}. Una \textit{cover} altro non è che un qualsiasi file multimediale, mentre lo stego-oggetto è il file risultante dall'incapsulamento del messaggio all'interno della cover. In questo scenario un sistema steganografico è quindi da dichiararsi non sicuro qualora l'osservatore, avendo a disposizione solamente un potenziale stego-oggetto, possa anche solo \virgolette{sospettare} la presenza di una comunicazione segreta.

	\section{Modelli di sicurezza Information-Theoretic}
	La maggior parte dei modelli di sicurezza proposti nella letteratura hanno un approccio \textit{\underline{I}nformation-\underline{T}heoretic} (IT) \cite{sec}. Un sistema crittografico è di tipo IT se la sua sicurezza deriva dalla teoria dell'informazione, che ha come assunzione principale la potenza di calcolo illimitata dell'entità attaccante. Un sistema così definito risulta quindi inattaccabile. La definizione di \textit{sicurezza incondizionata} di Z\"oller \cite{zoller}, per esempio, modella l'insieme dei messaggi $M$, delle cover $C$, delle chiavi $K$ e dei possibili stego-oggetti $X$ come variabili aleatorie. Sotto queste condizioni un sistema steganografico è considerato sicuro se la conoscenza delle distribuzioni di probabilità delle variabili aleatorie $X$ e $C$ non rivela alcuna informazione circa $M$.\\Un altro approccio \cite{approach} usa come metrica della sicurezza l'entropia relativa fra $X$ e $C$, $E(C\Vert X)$. L'entropia relativa fra due distribuzioni è un numero non-negativo, ed è uguale a $0$ se e solo se le due distribuzioni sono uguali. Pur non essendo l'entropia relativa una distanza nel senso matematico del termine -non è simmetrica e non rispetta la disuguaglianza triangolare- è utile pensare ad essa come una funzione che restituisce un valore interpretabile come la distanza fra due distribuzioni. Un sistema è quindi considerato sicuro secondo questa definizione se $E(C\Vert X) \leq \epsilon$, con $\epsilon$ piccolo a piacere. Intuitivamente minore è la \virgolette{distanza statistica} fra le cover e gli stego-oggetti, maggiore è la sicurezza del sistema. Si parlerà dunque di sistema steganografico \textit{perfetto} se $\epsilon = 0$. Questi modelli hanno una visione più vicina alla crittografia, dove si cerca di nascondere il \textit{contenuto} di un messaggio, piuttosto che la sua \textit{esistenza}, come nella steganografia. In sintesi, oltre ad un approccio inappropriato, i modelli IT risentono di almeno altri quattro grossi problemi di applicabilità:
		\begin{itemize}
		\item[i.] La costruzione di sistemi sicuri secondo tali definizioni non è affatto semplice.
		\item[ii.] I modelli in cui le possibili cover $C$ sono variabili aleatorie presuppongono la conoscenza della distribuzione di probabilità di tali variabili; ciò è quasi impossibile nella pratica. Sebbene in alcuni casi siano disponibili dei modelli approssimati, estrarre da questi una distribuzione di probabilità affidabile potrebbe essere computazionalmente arduo, o addirittura infattibile.
		\item[iii.] Qualora si riuscissero a trovare modelli che approssimano le distribuzioni di probabilità delle cover, potrebbe succedere che le modifiche introdotte in una cover da parte di un algoritmo siano minori dell'errore di approssimazione, rendendo così tale approssimazione inutile.
		\item[iv.] I modelli IT assumono che l'osservatore abbia a disposizione potenza di calcolo e memoria illimitati, condizioni non realistiche.
		\end{itemize}
		Alla luce di queste osservazioni sembrerebbe molto più ragionevole modellare la sicurezza di un sistema steganografico come un \textit{gioco probabilistico} fra l'osservatore attaccante e un giudice. L'entità attaccante, avendo la possibilità di osservare la comunicazione ed essendo a conoscenza del sistema steganografico in uso, deve essere in grado di discernere qualora l'oggetto consegnatogli dal giudice sia una cover o nasconda effettivamente un messaggio segreto.
		\section{Sicurezza condizionale di un sistema steganografico} 
		Prima di dare una definizione formale di un sistema steganografico sicuro basata sull'approccio condizionale verranno presentate una serie di definizioni necessarie alla sua comprensione.
				\begin{defn}
				Siano $C$ l'insieme delle possibili cover ed $M = \lbrace 0,1 \rbrace ^*$ l'insieme dei possibili messaggi. Per entrambi gli insiemi è richiesto solamente che siano generabili da un algoritmo probabilistico in tempo polinomiale. Un sistema steganografico può essere definito come una tripla di algoritmi polinomiali probabilistici:
				\[S=\langle G,E,D\rangle\]
dove:
			\begin{itemize}
			\item $G: 1^n \rightarrow k \in \lbrace 0,1 \rbrace ^n$, è un algoritmo che modella la generazione della chiave. 
			\item $E$ rappresenta il processo di incapsulamento del messaggio. Dati in input una cover $c \in C$, un messaggio $m \in M$ e la chiave $k$ generata da $G$, restituisce uno stego-oggetto $s \in C$.
			\item $D$ infine è il processo di decodifica. Dati in input lo stego oggetto $s$ e la chiave $k$ restituisce un messaggio $m' \in \lbrace 0,1 \rbrace ^*$. Se in $s$ vi era effettivamente nascosto un messaggio $m$, allora $m' = m$.
			\end{itemize}
		\end{defn} 
		\noindent Da questa definizione segue che un eventuale osservatore attaccante, qualora volesse rilevare il contenuto steganografico in una comunicazione, dovrebbe risolvere il problema dei prigionieri, che nella steganografia prende il nome di \textit{problema della decisione steganografica} ed è definito come segue:
	\begin{defn}
	Dato $s \in C$, si dice \textit{problema della decisione steganografica} trovare, se esistono, una chiave $k \in \lbrace 0,1 \rbrace ^*$ ed un messaggio $m \in M$ tale che $D(s,k) = m$. 
	\end{defn}
	\noindent Nel gioco probabilistico l'attaccante deve avere la possibilità di osservare la comunicazione e di conoscere il sistema steganografico in uso, questo è possibile grazie alla presenza di due \textit{oracoli}. Il primo oracolo memorizza il numero di richieste e restituisce la prossima cover della successione:
		\begin{defn} Sia $C$ un insieme di cover generabili da un algoritmo probabilistico polinomiale, si dice \textit{oracolo steganografico} $U$ una successione infinita di cover $c_1, c_2, \cdots, $ tutte disegnate dall'insieme $C$.
	\end{defn} 
	\noindent Il secondo oracolo ha invece il compito di permettere l'esplorazione completa del sistema steganografico in uso:
			\begin{defn} Siano $\langle G, E, D \rangle$ un sistema steganografico e $k \in \lbrace 0,1 \rbrace ^*$, con $k$ generata da $G$. Si dice un \textit{oracolo di valutazione strutturale} $V_k$ un algoritmo tale che dati in input un messaggio $m$ e una cover $c$, restituisce un oggetto $s$ tale che $E(c, m, k) = s$ e $D(s, k) = m$.
	\end{defn}  
	\noindent In definitiva quest'ultimo oracolo deve avere l'abilità di restituire uno stego-oggetto contenente uno specifico messaggio, senza conoscenza della stego-chiave $k$.\\ Prima di procedere con l'esposizione dei passi del gioco probabilistico è utile fornire al lettore la seguente definizione.
	\begin{defn}
	\label{trascurabile}
	Una successione $\lbrace n_i \rbrace$ di numeri reali non negativi si dice \textit{trascurabile} se 
	\[\forall \, \mathit{polinomio} \, p \: \exists i_0 \in \mathbb{N} \, \vert \, \forall i \geq i_0 \, : \, n_i < 1/p(i) \]
	\end{defn}
	\noindent Intuitivamente il gioco probabilistico procede in questo modo: l'attaccante $A$ interroga ripetitivamente entrambi gli oracoli. Il primo oracolo gli permette di \virgolette{osservare} la comunicazione, mentre il secondo gli dà la possibilità di costruire stego-oggetti. Una volta che il ragionamento di $A$ è terminato, il giudice, con probabilità $1/2$, consegna ad $A$ uno stego-oggetto o una cover semplice. Se $A$ riuscisse ad avere un \textit{vantaggio sistematico} nella decisione il sistema steganografico non sarebbe da considerarsi sicuro.\\ Verrà ora data una definizione formale dei passi del gioco probabilistico : 
	\begin{itemize}
	\item \textbf{Passo 1} Il giudice esegue $G(1^{n})$ per costruire la stego-chiave $k$ di lunghezza $n$ e dà accesso all'attaccante ad un oracolo di valutazione strutturale $V_k$ che simula l'algoritmo $E$ con chiave $k$ in input. 
	\item \textbf{Passo 2} L'attaccante esegue una serie di computazioni polinomiali interrogando $V_k$ e $U$. $V_k$ viene interrogato con $n_1$ messaggi e cover, restituendo in output i corrispondenti stego-oggetti $s_1, \cdots , s_{n_1}$. Gli stego-oggetti soddisfano $\forall i \in [1, n_1] \, : \, E(c_i, m_i, k) = s_i \, \wedge \,  D(s_i, k) = m_i$. $U$ viene invece interrogato $n_2$ volte restituendo le corrispondenti cover $c_1,\cdots , c_{n_2}$. L'unica restrizione è che la complessità totale in tempo sia polinomiale. Si noti inoltre che l'input per $V_k$ non deve necessariamente essere generato da $U$.
	\item \textbf{Passo 3} Una volta che l'attaccante ha terminato il passo 2, il giudice interroga $U$ due volte e preleva due cover $c_1,c_2 \in C$. Successivamente sceglie un messaggio $m$, calcola $s = E(c_2, m, k)$ e con probabilità $1/2$ consegna all'attaccante $c_1$ o lo stego-oggetto $s$.
	\item \textbf{Passo 4} A questo punto l'attaccante esegue un test probabilistico per decidere se l'oggetto passatogli dal giudice sia uno stego-oggetto o una cover senza alcun contenuto segreto e pubblica la sua congettura. 
	\item \textbf{Passo 5} Il sistema steganografico può essere definito sicuro per l'oracolo $U$ se il \textit{vantaggio} dell'attaccante è trascurabile secondo la definizione \ref{trascurabile}, dove il vantaggio è definito come $P(congettura \; corretta) - 1/2$.
\end{itemize}
Siamo ora in grado di dare una definizione formale di sistema steganografico sicuro rispetto ad un insieme di cover $U$.
\begin{defn}[\textsc{U-Sicurezza}] \label{sistemaSicuro} Sia $S=\langle G,E,D \rangle$ un sistema steganografico che opera su un insieme \textit{finito} di cover $C$ t.c $\forall c \in C \, : \, \Vert c \Vert \leq n $ con $\Vert c \Vert$ lunghezza della cover in bit ed $n$ costante. Inoltre siano $U$ un oracolo steganografico, $V_k$ un oracolo di valutazione strutturale e $k \in \lbrace 0,1 \rbrace ^*$ una stego-chiave generata da $G(k')$. $S$ può essere definito \textit{U-sicuro} se il \textit{vantaggio} di un attaccante nel passo 5 del gioco probabilistico è una \textit{successione trascurabile} $p(k')$ rispetto a $k'=\Vert k \Vert$. Il vantaggio, e quindi $P(congettura \; corretta)$, è calcolato su tutte le chiavi $k$ e tutte le scelte nel passo 3 del gioco probabilistico.  
\end{defn}
\noindent Da questa definizione segue quella di sistema steganografico \textit{probabilisticamente} sicuro.
\begin{defn}[\textsc{Sicurezza probabilistica o condizionale}] \label{sistemaSicuro1}  Un sistema steganografico $S=\langle G,E,D \rangle$ è \textit{probabilisticamente sicuro} per un insieme di oracoli $\mathcal{O}$, se $\forall \, U \in \mathcal{O} \, : \, S$ è \textit{U-sicuro}.
\end{defn}
\noindent Sebbene la sicurezza condizionale non implichi quella teoretica propria dell'IT, le definizioni \ref{sistemaSicuro} e \ref{sistemaSicuro1} sono utilizzabili e significative se applicate a casi reali \cite{sec}. 	  