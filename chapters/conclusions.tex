% !TEX root = cag-reviewernet.tex

\chapter{Conclusions}
\label{sec:conclusions}

We have presented ReviewerNet, a novel system for choosing reviewers by visually exploring scholarly data. ReviewerNet enables scientific journal editors and members of IPCs to search the literature about the topic of a submitted paper, to identify experts in the field and evaluate their stage of career, and to check possible connections with the submitting authors and among the reviewers themselves. This helps to avoid conflicts and to build a fairly distributed pool of reviewers. To do so, ReviewerNet features a combined visualization of the literature, the career of potential reviewers, their conflict of interests, and their nets of collaborators. Interestingly enough, the system is able to help the process even without exploiting any content-based analysis of the papers.

%The results from a user study involving 15 senior members from the Computer Graphics community confirmed that they were able to get acquainted with the system even with a very limited training, and  appreciated the different functionalities of ReviewerNet and its capability of improving the reviewer search process.

The evaluation of a preliminary version of the demonstration platform with both in-house testers and members from the Computer Graphics community confirmed that the users were able to get acquainted with the system even with a very limited training, and  appreciated the different functionalities of ReviewerNet and its capability of improving the reviewer search process.  
Some of the users also highlighted the potential of ReviewerNet as a tool to support bibliographic research, besides the reviewer selection process. 

The evaluation also highlighted that there was room for improving the system, which we did by: improving on the effectiveness of the visualization and the user-friendliness of the interaction modes; improving the initial process of selecting key papers, whose manual insertion was signalled as a weakness by some users, by importing the bibliography of papers; defining a procedure to generate instances of the platform with customizable data coverage, by loading one's own reference dataset. Additional directions for improvement include: an automatic strategy to suggest key papers by computing network features (e.g., betweenness  centrality); and a user-friendly visual interface for building the reference dataset, through the assisted selection a list of venues of interest.   

Also, we are currently planning to carry out a second formal user study with end users on the revised platform, to assess both the individual visualization design strategies, and the overall system usability and performance.

%, in its current state ReviewerNet focuses on the deterministically mechanical part of the process, minimizing the possibility of introducing any bias in the process. 
%We also plan to discuss with providers of scholarly data about the possible release of a version of ReviewerNet with customizable data coverage, to be used by the various scientific communities.  

