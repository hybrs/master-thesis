% !TEX root = main.tex
\chapter{Conclusions}
\label{sec:conclusions}

We have presented ReviewerNet, a novel system for choosing reviewers by visually exploring scholarly data. ReviewerNet enables scientific journal editors and members of IPCs to search the literature about the topic of a submitted paper, to identify experts in the field and evaluate their stage of career, to check possible connections with the submitting authors and among the reviewers themselves, helping therefore to avoid conflicts and to build a fairly distributed pool of reviewers. To do so, ReviewerNet features a combined visualization of the literature, the career of potential reviewers, their conflict of interests, and their nets of collaborators. 

The results from a user study involving 15 senior members from the Computer Graphics community confirmed that they were able to get acquainted with the system even with a very limited training, and  appreciated the different functionalities of ReviewerNet and its capability of improving the reviewer search process.

Interestingly enough, the system is able to help the process even without exploiting any content-based analysis of the papers. While it is true that there is room for improving the system by partially automating the choice of the key papers, in its current state ReviewerNet focuses on the deterministically mechanical part of the process, minimizing the possibility of introducing any bias in the process. 

In the future, we plan to discuss with providers of scholarly data about the possible release of a version of ReviewerNet with customizable data coverage, to be used by the various scientific communities.  

